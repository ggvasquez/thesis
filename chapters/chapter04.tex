\section{NA64 experiment} 
The NA64 experiment is a fixed-target experiment at the CERN SPS combining the
active beam dump and missing energy techniques to search for rare events.\\ A
fully hermetic detector placed on the H4 beam line has been built with the
primary goal to search for light dark bossons (Z') from dark sector that are
coupled to photons, e.g. dark photons (A'), or sub-GeV Z' coupled only to
quarks. In some cases the Z' is coupled only to $\mu$ or tau, so we call the Z'
the dark leptonic gauge boson. The experiment is also capable to search for $K_L
\rightarrow $invisible decay, which is complementary to $K+\rightarrow \pi^+ +
\nu \nu$, and invisible decays of $\pi_0$, $\eta$,$\eta'$, $K_S$ mesons.\\ The
advantage of this approac is that the sensitivity (or number of signal events)
of the experiment is roughly proportional to the Z' coupling squared
$\varepsilon^2$, associated with the Z' production in the primary interaction in
the target/ While in a classical beam dump experiment, it is proporiotional to
$\varepsilon^4$, one $\varepsilon^2$ came from the Z' production, and another
$\varepsilon^2$ is either from the probability of Z' decays or their
interactions in a detector located at a large distance from the beam dump.\\ The
sensitivities of these two methods depend on the region under study in the
($\varepsilon^2$,$m_Z$) parameter space, background level for a articular
process, available beam intesity, etc.\textcolor{red}{[Beam intensity]} In some
cases, much less running time and primary beam intensity are required to observe
a signal event with our approach.\\


\subsection{Physics Motivation}
Standar Model.\\
g2 muon anomaly\\

\subsection{Dark Photon signal}
U(1) broken symetry --> massive dark photon\\
type of mixing
coupling constant and subGeV mass connect with g2 muon anomaly.
How to detect him?

\subsection{Particle identification}

\subsection{Setup}
\section{Synchroton Radiation Detector }
\subsection{BGO}
\subsection{Pb+Sc}
\subsection{LYSO}
\section{Calibration}
\section{Position and time resolution}
\section{Hadron rejection}
\section{Purity 100 GeV electron identification}
\section{Summary}


