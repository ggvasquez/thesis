\section{Introduction}
The aim of this chapter is to show the characteristics of the new detectors present on the upgrade for ATLAS detector,
and how can achieve the requirements for the high luminosity events.
The results of each test to characterize the detector are present.


\subsection{High Luminosity Large Hadron Collider}
The Large Hadron Collider (LHC), run by CERN at the Franco-Swiss border near Geneva, is a circular accelerator with 27km of acceleration pipes, is the largest scientific instrument ever designed and built for scientific research.
Successfully commissioned in March 2010 for proton-proton collision with a \SI{7}{GeV} center-of-mass energy.\\ The LHC is
pusshing the limits of human knowledge, enabling physicist to go beyond Standar Model (SM): the enigmatic Higgs boson,
mysterious Dark Matter and the world of supersymetry are just three of the long-awaited mysterous that the LHC will
unveil. The announcement given by CERN on 4 July 2012 about the discovery of new boson at 125-126 GeV, almost certainly
the long awaited Higgs particle, is the first fundamental discovery, hopefully the first of a series, that the LHC can
deliver.\\ Such discovery was thanks to the different detectors located on the four interaction points; ALICE, LHCb, CMS
and ATLAS. This last one is the detector where our university is taking part.\par
\par


% -------- LHC Schedule Table ------ 
\begin{table}[h]\footnotesize
\centering
\begin{tabular*}{0.8\textwidth}{rcccc}
\cellcolor{blue} &\cellcolor{blue}\textcolor{white}{Period} &\cellcolor{blue}\textcolor{white}{Energy $\sqrt{s}$} &\cellcolor{blue}\textcolor{white}{Luminosity ${\cal
L}$} &\cellcolor{blue}\textcolor{white}{Integrate ${\cal L}$} \\
\cellcolor{cyan} \textcolor{white}{Run I} 	& 2010-2012 & 7-8 TeV & \SI{6e33}{cm^{-2}s^{-1}} & \SI{25}{fb^{-1}}\\
\cellcolor{cyan} \textcolor{white}{LS1} 		&\cellcolor{lightgray}2013-2014 & \multicolumn{3}{l}{\cellcolor{lightgray}Go to design energy, nominal luminosity,
bunch spacing 25ns}\\
\cellcolor{cyan} \textcolor{white}{Phase 0} & 2015-2018 & 14 TeV & \SI{1e34}{cm^{-2}s^{-1}} & \SIrange{75}{100}{fb^{-1}}\\
\cellcolor{cyan} \textcolor{white}{LS2} 		&\cellcolor{lightgray}2019-2020 & \multicolumn{3}{l}{\cellcolor{lightgray}Upgrade muon spectrometer;NSW,LAr
Calorimeter \& FTK}\\
\cellcolor{cyan} \textcolor{white}{Phase 1} & 2021-2023 & 14 TeV & \SI{2e34}{cm^{-2}s^{-1}} & $\sim$\SI{350}{ fb^{-1}}\\
\cellcolor{cyan} \textcolor{white}{LS3} 		&\cellcolor{lightgray}2024-2025 & \multicolumn{3}{l}{\cellcolor{lightgray}New Inner Trackker and trigger
architecture}\\
\cellcolor{cyan} \textcolor{white}{Phase 2} & 2026-2030 & 14 TeV & \SI{5e34}{cm^{-2}s^{-1}} & $\sim$\SI{3000}{fb^{-1}}\\
\end{tabular*}
\caption{LHC Schedule }\label{lhcschedule}
\end{table}


\subsection{ATLAS Detector}
The ATLAS detector it is a general-purpose detector, designed to explore proton-proton colissions at center of mass up
to $\sqrt{s}=$14GeV at the Large Hadron Collider (LHC) at the European Laboratory for Particle Physics (CERN). Looking
for understand the foundations of matter and forces, in particular the nature of mass in a broad physics programme,
	including the capability of discovering the Higgs boson over a wide mass range and performing searches for the
	production of heavy particles that would indicate phyics beyond the standard model, such as SUSY particles, as well as
	searches for other massive objects. \par 
To be able to detect such important and rare events this machine has mutliple and complex detector systems. In the
central part of this cylinder we have the Inner Detector, the first part of ATLAS to see the decay products of the
collisions, so it is very compact and hihgly sensitive. it consiste of three different systems of sensors all immersed
in a magnetic field parallel to the beam axis. The {\bf Inner Detector} measures the direction, momentum, and charge of
electrically-charged particles produced in each proton-proton collision. The next part is the Calorimeter (red and green
on figure \ref{fig:ATLAS}), which  measure the energy of a particle when loses its energy as it pass through the
detector. It is usually designed to stop entire or ``absorb" most of the particles coming form a collision, forcing them
to deposit all of their energy whithin the detector. Calorimeters typically consist of layers of ``passive" or
``absorbing" high-dense material -for example, lead- interleaved with layers of an ``active" medium such as solid
lead-glass or liquid argon.\par
\begin{figure}[ht]
		\centering
		\includegraphics[width=0.7\textwidth]{ATLAS.pdf}
		\caption{ATLAS detector, Muon Spectrometer(in blue)}\label{fig:ATLAS}
\end{figure}
Electromagnetic calorimeters measure the energy of electrons and photons as they interact with matter. Hadronic
calorimeters sample the energy of hadrons (particles that contain quarks, such as protons and neutrons) as they interact
with atomic nuclei. The components of the ATLAS calorimetry system are: the {\bf Liquid Argon (LAr) Calorimeter} and the
{\bf Tile Hadronic Calorimeter}.\par
Calorimeters can stop most known particles except muons and neutrinos. Muons are particles that usually pass through the
Inner Detector and Calorimeter undetected, they can penetrate through large amount of material without any strong
interaction, they have long lifetime hence offering lepton decay channels for heavy objects as:

\par

For those particles the ATLAS detectors has the {\bf Muon Spectrometer}, made up of 4.000 individual muon chambers
(drift
chambers) is in charge of identify each one of this muons. This last detector system is a key point to help us finding
new physics, thanks that this muons can be considered as a stable particles whithin the volumen range of the ATLAS
detector, one can search for products from heavy particles decays to leptons, identifiying properly  its momentum. Well
that part is provided by the {\bf Magnet System}, made of three sections; the {\bf Central Solenoid Magnet} with 2T
magnetic field bends the
charged particles for momentum measurement near the interaction points, helping the Inner Tracker system, the {\bf
Barrel Toroid} bends the particles with low tranversal momentum, and the {\bf Endcap Toroid} with 4T take part of the
high tranverse momentum.\par

\subsubsection{Coordinate system}
A common coordinate system is used through ATLAS. The interaction point is defined as the origin of the coordinate
system. The z-axis runs along the beam line. The x-y plane is perpendicular to the beam line and is referred to as the
tranverse momentum, $p_T$. The positive x-axis points from the interaction point to the de center of the LHC ring; the
positive y-axis points upward to the surface of the earth. The detector half at positive z-values is referred to as the
``A-side", to the other half the ``C-side". The transverse plane is often described in terms of $r-\phi$ coordinates. The
azimutal angle $\phi$ is measured from the x-axis, around the beam. The radial dimension, $r$, measures the distance
from the beam line. The polar angle $\theta$ is defined as the angle from the positive z-axis. The polar angle is often
reported in terms of pseudorapidity, defined as $\eta = -\ln \tan (\theta/2)$. The distances $\Delta R$ is defined in
$\eta-\phi$ space as $\Delta R = \sqrt{\Delta \eta^2 + \Delta \phi^2}$.\par




%\textcolor{red}{Such energy has been achived from 2015 and successfuly working with a
%luminosity of $\mathcal{L}=$\SI{1e34}{cm^{-2}s^{-1}} from 2016.}\\
%\subsubsection{}

\subsubsection{Detector Upgrade}

In manner to fullfill the LHC program (in table \ref{lhcschedule}), and in order to benefit from the expected high luminosity
performance that will be provided by the Phase-I upgraded LHC, the first station of ATLAS muon end-cap system (Small
Wheel, SW) will need to be replaced.  The {\bf New Small Wheel (NSW)} will have to operate in a high background radiation
region (up to \SI{15}{kHz/cm^2} of photons, and \SI{75}{Hz/cm^2} of neutrons) while reconstructing muon tracks with high precision as well as furnishing information for
the Level-1 trigger. These performance criteria are demanding. In particular, the precision reconstruction of tracks for
offline analysis requires a spatial resolution about 100\micro{m}, and the Level-1 trigger track segments have to be
reconstructed online with an angular resolution of approximately 1mrad. \par
\begin{figure}[ht]
		\centering
		\includegraphics[width=0.7\textwidth]{NSW.png}
		\caption{New Small Wheel}\label{fig:nsw}
\end{figure}

The NSW will have two chamber technologies, one primarily devoted to the Level-1 trigger function (small-strip Thin Gap
Chambers, sTGC) and one dedicated to precision tracking (Micromegas detectors, MM). The sTGC are primarily deployed for
triggering given their single bunch corssing identification capability. The MM detectors have exceptional precision
tracking capabilities due to their small gap (\unit{5}{mm}) and strip pitch (approximately 0.5\si{mm}). Such a precision is
crucial to maintain the current ATLAS muon momentum resolution in the high background environment of the upgraded LHC.
The MM chambers can, at the same time, confirm the existence of a track segments found by the muon end-cap middle
station (Big Wheels) online. The sTGC also has the ability to measure offline muon tracks with good precision, so the
sTGC-MM chamber technology combination forms a fully redundant detector system for triggering and tracking both for
online and offline functions. This detector combination has been designed to be able to also provide excellente
performance for the eventual High Luminosity LHC upgrade.\par 



\section{small-Strip Thing Gap Chamber}

The small-Strip Thin Gap Chamber (a.k.a sTGC) detector it is a multi-wire proportional chamber (MWPC), a detector type
with a relativetely old technology, its succesfuly introduction to detector system in 1968 gave the
Nobel prize to George Charpak in 1992. 
Those devices has been a major ingredient in detector system since they can achieve spatial resolution of several
microns, and has typical time resolution of about 50ns.
The sTGC has been design to exploit these features,  working with a cathode-anode pitch smaller than the anode-anode pitch, mostly based on the design of the Thin Gap
Chamber\cite{tgc}, with thinner strips as the main improvement from the previous version. The TGC tecnology has been
used since 1988 in OPAL experiment and currently are part of the the muon spectrometer in ATLAS. \\ This new  chamber
has the advantage of having a \unit{3.2}{mm} strip-pitch (\unit{2.7}{mm} strips and \unit{0.5}{mm} gap) compare to the \unit{6}{mm} from the previous TGC, that is
why the  small strip prefix. Chambers with different strips sizes has been build and test under pion beams,
chosed the \unit{3.2}{mm} as the best option\cite{stripwidth} to provide a resolution better than \unit{100}{\micro m}.
This change will improve the measurement of charge centroid position by charge interpolation.\par
To improve the time response, the cathode surface resisistivity has been reduce by
a factor 10, to reduce charge accumulations on the cathode when chamber operates at high rate, lowering from 1
M$\Omega/\square$ to 100-200k$\Omega / \square$ resisitivity on the graphite layer. At the same time cathode-readout
plane (strips or pads) discance has been reduce to \unit{0.1}{mm}(\unit{1.6}{mm} before) to increase the capacitive coupling by 10, therefore the RC factor
keeps unchanged
 \par
%----------- sTGC mode picture -----

\begin{figure}[ht]
		\centering
		\includegraphics[width=0.5\textwidth]{sTGC_layout.png}
		\caption{Single plane sTGC}\label{fig:sTGC}
\end{figure}

The sTGC is made of two resistive cathodes planes, with copper readout plane with strips on one layer and the other one
with pads, with \unit{8x12}{cm$^2$} area used for fast pattern recognition of tracks to select strips for read out. A big
advantage compare to the TGC, which has not this feature.\par
The cathodes are made of FR4 with \unit{1.6}{mm} of thickness, where  \si{100}{\micro m} of copper is etched for strips (pads) and then
pressed with a \unit{100}{\micro m} of FR4 over it and then sprayed with graphite to provide  for
TGC) so it can be treatead as a resistive cathode plane.\\ The anodes are golden tungsten wires of \unit{50}{\micro m} diameter,
distributed at \unit{1.8}{mm} between each other and a gas gap of \unit{2.8}{mm}. To work with such geometry several test were made to find the proper gas
mixture\cite{gaschoice}. Find the most suitiable mixture of  55\% well known carbon dioxide(CO$_2$) as a quenching gas, and a
45\% of n-pentane(n-C$_5$H$_{12}$), which allow it to work in a limited proportional region thanks to its capability to absorb UV photons
due to its many ways of molecules degree of freedom, vibrational, rotational, etc.\par 
\textcolor{red}{relatively low sensitivity to mechanical variations.[ref Mechanical variations]}\par  

%----------- How its produce the ionization and drifts  -----



\subsection{Electric field simulation}


math explanation of gain dependence of electrical field
show countourplot for 5 wires and percentage of high electric field...
blind spots!

\subsection{Timming properties}
it has a fast trigger within 25ns of first electrons arrivals.

\subsection{}


\section{Construction process}
Since the main novelty on this detector is the high resolution obteined on xy plane due to the strip boards and the
alignment between each chamber to get precission about 50\micro m and 30\micro m respectively, its important to discuss
how we achieve those numbers. Everything relies on how well are those chamber built and also how the cathodes boards
(strips and pads) are fabricated. The size of each chamber are around 1m$^2$ to 2m$^2$ and for the current fabricant it
is complicated to achieve 50 \micro m resolution etching a pcb board accross 1.5 m without taking into consideration
that the standard size for pcb boards are 70 cm long.  The attempt of this sections is to give an idea how the sTGC
Quadruplets are built, mostly on the first module 0 produce by UTFSM, which is the QS1 (Quadruplet Small sector, part
1).  Being the smallest detector to be produce for the NSW, has some pro and cons. The main cons is related to the
posistion of the QS1 inside the NSW, it is the closest one to the interaction point and for that it get the highest rate
of particles. For the same reason the position resolution is a key point and the response against high rate particles it
is a must.  Some pros are related to the size of it, with approximetely 1m long and 55 cm the small side and 75 cm the
large side of the trapezoidal shape, the sTGC QS1 can be handly without any problems, its 45 kg at the final step is
easy to manage between two people, and with almost 1m long, the etching of strips boards (2.7 mm strip width and 0.5 mm
etch) can be achieve with 50 \micro m. 

\subsection{Quality Control of cathode boards}

The cathodes for the module0 were maid by an Italian company MDT, and since it was the first production, the review was
done at their place.\\ The thickness of the board is measured in 19 points around the perimeter with a micrometer. The
values of theses measurments must be whithin 1.6 mm $\pm$25\micro m,exceding this numbers lead to the partial rejection
of the cathode boards, however if there is a single point deviation of less than $\pm$ 35\micro m about the average, it
could be used in combination with another cathode board that does not have the same local deviation. The raw data is
found in appendix X.\par An electrical test is donde with a multimiter, to check if there is any short between strips or
pads depending on the cathode board.\par The last step and the most important is the dimensional control: this must be
perform on a flat surface (done on a granite table at the construction site), with 2 pins that match the brass inserts
on the cathode and using a special caliper above the cathode board the misalignment is measured. The caliper is an
aluminuim ruler machining  with a precision of 30 \micro m at 20 Celsius degrees  has the same strip pitch for the first
and last five strips and to avoid any paralax the thickness is the edge for those strips is 1mm. Looking with a lens
glass around this point it posible to detect some misaligment between this two strips (caliper and cathode board). A
photography is taken and analize to calculate this misaligment. For such distance (about 1 m long) some precaution must
be take, considering the expansion coeficient for both material.  machining with a precision of 30\micro m measure at 20
Celsius degrees.\par

%------- Photography strip cathode board measurement ------------ % 
\begin{figure}[ht]
	\hfill
	\centering
	\begin{subfigure}[b]{0.35\textwidth}
		\centering
		\includegraphics[width=\textwidth]{alignment.png}
		\caption{Al ruler used to check shift over the last strips.}
		\label{fig:ruler}
	\end{subfigure}
	\hfill
	\begin{subfigure}[b]{0.35\textwidth}
		\centering
		\includegraphics[width=\textwidth]{zoom_in.png}
		\caption{Zoom-in Comparing strip position}
		\label{fig:zoom}
	\end{subfigure}
	\hfill
\end{figure}

\subsection{Cathode preparation}
Once the cathodes pass all the dimensional control are cleaned with Acetone and Isopropyl acohol and placedplaced on a
granite table with a flatness of better than 30\micro m, and a vacuum system underneath and fixed on the edges with
metal jigs which also has marks for the internal wire support or chamber section.  The places which does not be sprayed
with graphite, like the wire support and the edges are covered with a \unit{3.5}{m m} black tape to  the designated wire
support locations across the board, and a blue tape the edges preventing spray graphite on the places where they will be
glue. 

\subsection{Graphite spraying}
A key point for this process is to prepare the 'painting' a mixture of Graphite-33 with Plastik-70 bonding agent. The
graphite must be agitated for at least 2 hours before mixing with Plastik-70. A proper ratio of 1500g Graphite and 540g
Plastik is mixed for 20min before spraying.\par A spraying machine is in charge to get this process done, meanwhile a
temperature and humidity must be controlled. After the cathode is painted the superficial resistance is measured on the
edges and values above \unit{100}{k$\Omega/\square$} must get otherwise the cathode needs to be sprayed again.\par


\subsection{Polishing}
In manner to ensure an uniform resistivity across the chamber, the cathode is visually divide in 5x6 squares. Inside
each square t e resistivity is measure in 5 to 7 points with a probe and simultaneously brush it in the orientation that
the wires will be wound. The brush must be done carefully without over-polish areas because onces resistance drops down
nothing will bring it back up. 


\subsection{Glue internal parts}
After removing all the blue and black tape, all the internal parts such buttons to provide mechanical support to the gas
gap, the wire support which help them to not bend due to gravity and create catenary effect, and all the external frames
to provide the \unit{1.4}{mm} height for the gas gap. All this part are cleaned with isopropyl alcohol, meanwhile the
glue, a type of epoxy (2011-Araldite) is prepare. This glue will not only fix the part, also will fill the surfaces
where those part are had less thickness than is requested.


\subsection{Winding wires} A flat table which can spin around on one axis is used to winding the cathodes board. On each
side of the table one cathode with all the internal parts previosly glued is mounted and tight it with metal clamps on
the edges, meanwhile a vacuum is apply underneath to ensure the flatness of the cathode.  A winding machine is in charge
of this process, taking all the precautions to place every wire at \unit{1.8}{mm} between each other with
\textcolor{red}{\unit{50}{\micro m} precision?}.\\ After the process is done, all the wires are solderd in group of 10
in the wire rulers with \unit{10}{M$\Omega$} resistor to the high voltage line, later on the remaining wire can be cut
and the metal clamps around the edges can be removed. 


\subsection{Detector Assembly}

Once the Pad cathode board with all wires soldered is cleaned with clean water and dry with clean air. The board is
place on the granite table, vacuum is applied underneath so it can be ready to connect it to high voltage. It is
necessary to monitor the current from the cathode meanwhile the voltage is increased, starting with 100V and reach
3000V, every 100V step the current is monitoring and check that never goes higher than \unit{1}{\micro A}, if it is the
case, the cathode needs to watch carefully on each corner to see some small sparks around dust or remaining glue from
the previous process.\\ Reaching the nominal current, the strip cathode board is placed against the pad cathode board
carefully, and an aluminuim frame with a sillicon rubber is placed on top to isolate the chamber from the environment,
afterwards the vacuum is applied to this chamber and only CO2 is flushing inside the chamber. Now it is the time to turn
on the power supply and watch if there is no sparks (monitoring the current), if it is not the case, the glue is
prepared to close the chamber inmediataly to prevent any dust enter the chamber when the sillicon ruber and the strip
cathode board is remove. Finishing this process, a \underline{single chamber detector is considered done}.\\ Closing two
chambers is possible build a doublet, for that it is necessary to glue a honeycomb with a well known thickness
(\unit{5$\pm$0.02}{mm}) 


\subsection{Planarity and thickness measurements}
Total thickness.

\section{Gain uniformity measurements}

After the chambers are built it is important to look for any malfunctioning. To check the behaivour of the detector without
any electronic readout attach to strips or wires, is to use a radiation source and move it acrros the de
sensitive area and expect the same behaviour current draw from it. This will give us a the first answer of the gain of
our detector.\par
In this test it is important to understand what can produce variation on the gain. There are two ingredients that can produce gain variations on wire detectors,
the first one is the \'nature" gain fluctuations from the charge production in proportional counters which follow Polya
distribution, however is less pronounced in limited-proportional mode such as sTGC working region.\\
The second one is related to the mechanical tolerances, this part is very well known since 40 years as it is presented
on Sauli's book\textcolor{red}{Insert ref} about drift chambers and tell us:
\begin{itemize}
\item A diameter variations of the wire about 1\% (fabrication precision) will result on a 3\% change in the gain.
\item  \unit{100}{\micro m} difference in the gas gap thickness(\unit{2.7}{mm}) results in about 15\% change on the gain.
\item The effect of a wire displacement of about \unit{100}{$\mu$m}
of a wire plane results in 1\% in the charge of the two adjacent wires which with a gain of $\sim10^6$ will give a
$\sim10\%$ change on the gain.
\end{itemize}
Taking all of this in consideration is expected to get a gain variation less than 20\% as Quality Acceptance on the
Construction manual\textcolor{red}{Insert Ref}.\par
In this test the gain is considered as the current draw measured from the power supply and its need to test under two different working points (bias voltage), one when the chamber it is not in the limited
proportional region, 2500 volts, a take it as a reference compare to the 2900 volts which is the operational voltage.\par

%-------- why x ray source ? ----------
For such test the x-ray source is used thanks to the advantages:\par
\begin{itemize}[noitemsep, topsep=0pt, parsep=0pt, partopsep=2pt]
	\item Mostly mono-energetic photons.
	\item Variable current: which can provides different rates. [\unit{1}{$\mu$A} - \unit{200}{$\mu$A}]
	\item Variable voltage: modifying breaking voltage of electrons inside the x-ray gun. [\unit{10}{keV} - \unit{50}{keV}]
	\item Different spot size: with a set of collimator it is possible to irradiate only interesting area.
	\item Portable, it is possible to move across the sensitive area of the detector.
\end{itemize}


\subsection{Setup}
To perform such test a x-ray gun called Mini-x from Amptek is used, with silver (Ag) as transmission target with a
beryllium (Be) end window is used.  Working under \unit{50}{kV} with \unit{45}{$\mu$A}, the emission spectra
Fig.\ref{fig:minixgun} show two main photo peaks with 22 and 25 keV. The aperture emissions is about 120 degrees, so a
collimator of 5 degrees is placed to provide a spot size of \unit{4}{mm}.\\ The gun is mounted on a robot arm KUKA, to
move it across the detector irradiating a long the wires. Since the high voltage power supply give us a current average
per second, the vertical and horizontal step can be adjust to perform a spot size of 4$mm^2$ however the Mini-x device
had some overheat issues so a \unit{5}{cm} vertical step was chosen to perform a full irradiation test in less time. A
more suitable step would be \unit{1}{cm} to detect internal structures of similar sizes. The horizontal step was
\unit{1.5}{cm}, enough to change completely from group wires, although it will better to improve the granularity it is
not possible since the issue explained before.


\begin{figure}[ht]
	\centering
	\includegraphics[width=0.5\textwidth]{minix_ag_1.png}
	\caption{X-ray emission spectra from Mini-X gun}\label{fig:minixgun}
\end{figure}

The irradiation test is taken in approximately one hour and the four chamber are done at the same time, meanwhile the
division of each chamber are connected to the same HV channel. 




\subsection{Results}
The next two set of histograms shows the distribution of the current draw at two different voltage, one at 2500V to take
it as  reference since the cahmber can be consider with no gain (no proportional mode), and the second one the voltage
at what our chamber must be working 2900V.\\ On the Figure \ref{fig:2500V} an average of approximately \unit{200}{nA}
can be observe on the four layers, considering \unit{50}{nA} when no source is present (leakage current).  As previously
discuss the important part is to get how uniform are the chambers across the whole sensitivity area, even the places
where the internal parts are found, such a wire supports and buttons, which in that case a notoriously decrease of gain
(less current draw) is expected due to the lack of gas volume, and so no amplification.\\

On the Fig\ref{fig:2900V} the current draw for each single chamber is shown and an  uniformity less than 17\% is
obteind, calculated as rms percentage over  mean. 
\begin{figure}[t]
	\centering
	\includegraphics[width=0.8\textwidth]{uniformity_2500.pdf}
	\caption{Current from power supply at 2500V}\label{fig:2500V}
\end{figure}

\begin{figure}[hb]
	\centering
	\includegraphics[width=0.8\textwidth]{uniformity.png}
	\caption{Current from power supply at 2900V}\label{fig:2900V}
\end{figure}
\begin{figure}[hb]
	\centering
	\includegraphics[width=0.6\textwidth]{xy_scan.png}
	\caption{XY Scan Layer 3 sTGC Module \#0}\label{fig:xy_scan}
\end{figure}


\section{Stability under high count rate of $\gamma$-rays}

One of the key feature of this detector is that must be able to work under high particles flow
rates (\unit{15}{kHz/cm$^2$}), and the first step is
check weather the device or its electronic components can handle this high rate. For this purpose the module 0 was
placed inside of a High Radiation Facility at CERN called GIF++.
The instalation has a Cesium-137 as a gamma source with an activity of approximately \unit{17}{TBq} and different attenuation
filters can be applied to get the flux needed. 
To get the particle rate a small size sTGC was installed as a {\bf Monitor}, which has the electronic readout from each
wire group, equipped with the SONY card (\textcolor{red}{provide description}).

Our module 0 together with the monitor was placed at 1.3m distance from the source, both connected in serie to the same
gas line, the temperature and preasure was recored to keep in track the working voltage, where most of the time the
enviromental conditions were 25 Celsius degree and \unit{971}{mb}.\\

\begin{figure}[!hb]
	\centering
	\includegraphics[width=0.6\textwidth]{monitor.png}
	\caption{Monitor rate}\label{fig:monrate}
\end{figure}


\begin{figure}[!ht]
	\centering
	\includegraphics[width=0.5\textwidth]{rate_filters.png}
	\caption{Filters applied}\label{fig:filters}
\end{figure}
On the fig.\ref{fig:monrate} the event rate register with the monitor is shown at different attenuation filters from the
gamma source, where attenuation 1 means \unit{5}{TBq}. The
brown region is our working region, and the red dot lines are the limits where our chamber must working inside the ATLAS
cavern. On fig. \ref{fig:filters} the  event rate is shown applying the attenuation factors which the gamma filters
provide, so all the lines must join, however we do see some inefficency when the rate goes higher than \unit{20}{kHz/cm} (blue
line), but at the same time, our interest shoudl go on the red line (attenuation factor of 2.2) which gives us the
background level present on the upgrade. If we compare it with lowest rates, the efficiency is about 95\% at
\unit{2.9}{kV}.

%----------------- SPATIAL RESOLUTION ---------------


\section{Spatial Resolution}
In order to achieve the precision reconstruction of tracks (offline) with a spatial resolution of about
\unit{100}{\micro m} per sTGC layer, and fast trigger on the region of interest (ROI) with Pads, two test beams were done.\\
In the spring of 2014, the Weizmann Institute of Science in Israel built the first full-size sTGC quadruplet detector of
dimensions \unit{1.2x1.0}{m$^2$}. This prototype consists of four sTGC strip and pad layers and is constructed using the full
specification of one of the quadruplets to be used in the NSW upgrade (the middle quadruplet of the small sector).
The first test beam experiment took place at Fermilab with one goal in mind, determine the position resolution of a
full-size sTGC.\\par
\begin{figure}[!ht]
	\centering
	\includegraphics[width=0.5\textwidth]{telescope.png}
	\caption{\small Schematic diagram of the experimental setup at Fermilab and coordinate systems used. Three layers of silicon pixel sensors are positioned before and after the sTGC detector.The dimensions are not to scale.}\label{fig:telescope}
\end{figure}
EUDET pixel telescope were use as a reference for measure beam position, using the technology of 6 Minimum Ionizing MOS
Active Pixel Sensor (Mimosa26) detectors with $\approx 5 \mu$m position resolution. Three in front of the beam, and three
after the sTGC as is shown in fig.\ref{fig:telescope} with \unit{15}{cm} between them and \unit{64}{cm} between each arm.
Each Mimosa26 detector has an active area of \unit{2.24}{cm$^2$} made of CMOS pixel
matrix of 576 rows and 1152 columns with \unit{18.4}{\micro m} pitch.\par
A \unit{32}{GeV} pion beam was used at rate of \unit{1}{kHz} over a spot of \unit{1}{cm$^2$} giving to the sTGC a very
precise pion trayectory thanks to the EUDET telescope.
Event triggering was controlled by a custom Trigger Logic Unit (TLU). The TLU received signals from two
\unit{1x2}{cm$^2$}
scintillators placed in front and behind the telescope. The TLU generated the trigger signal that was distributed to the
telescope and the sTGC readout electronics, which consist of a first application-specific integrated circuit (ASIC)
called VMM1 which has the ability to read out both positive (strips, pads) and negative (wires) polarity
signals, on 64 individual readout channel. The VMM1 analog circuit features a charge amplifier stage followed by a
shaper circuit and outputs the analog peak value (P) of the signal.The readout of the ASIC is zero suppressed and thus
only peak values of channels with signals above a predefined threshold are read and at the same time the VMM1 may be
programmed to provide the input signal amplitude of channels adjacent to a channel above threshold (neighbour-enable
logic), which in case of the strips is strictly necesary due to the cluster size is about 3-5 strips.\par
The precise position of a charged particle traversing an sTGC gas volume can be estimated from a Gaussian fit to the
measured charge on adjacent readout strips (referred to as strip-clusters from here on). Given the strip pitch of
\unit{3.2}{mm}
and sTGC geometry, charges are typically induced on up to five adjacent strips. The spatial sampling of the total
ionization signal over a small number of readout channels means that a precise knowledge of each individual readout
channel baseline is necessary in order to achieve the best possible measured spatial resolution. The baseline of each
individual readout channel was measured by making use of the neighbour-enabled logic of the VMM1 and its internal
calibration system. Test pulses were sent on one readout channel with the neighbour-enabled logic on, and baseline
values were obtained by reading out the analog peak values of the two chan- nels adjacent to the one receiving a test
pulse.\par
The silicon pixel hit positions were then used for reconstructing straight three dimensional charged-particle tracks. A
track quality parameter was obtained for each fitted pion track based on the $\chi^2$ of the track-fit. A small value of the track quality parameter corresponds to a straight track and a cut on this parameter can therefore be used to mitigate
multiple scattering which are not considered in this analysis.
\subsection{Analysis Model}
\subsubsection{Pixel telescope analysis}
In this model the intrinsic position resolution is obteined comparing the extrapolated beam trayectory from the pixel detectors
wtih the measurements in each of the sTGC quad planes. Each layer is analyzed separately to reduce the effect of the
multiple scattering and only tracks with $\chi^2<10$ are considered for the same reason. From fig.\ref{fig:telescope}
you can see that the {\it y-axis} is defined perpendicular to the strips, therefore sTGC strip-clusters provide
measurements of the particle position in the y-direction($y_{sTGC}$).
The position resolution is directly related to the profile of induced charge on the strips. The particle position is
estimated from a Gaussian fit to the induced charge distribution on the strips. The neighbour-enabled logic of the VMM1
was used. Strip-clusters with induced charge in either 3, 4 or 5 adjacent strips are selected.\par
The pixel telescope tracks provide both coordinates, x$_{pix}$ and $y_{pix}$ at the position of the sTGC layer studied. The
spatial resolution measurement is obtained by fitting the residual distribution $y_{sTGC}$  and $y_{pix}$ with a Gaussian model.\par


\begin{figure}[ht]
	\centering
	\hfill
	\begin{subfigure}[b]{0.45\textwidth}
		\centering
		\includegraphics[width=\textwidth]{strip-pitch.png}
		\caption{The differential non-linearity for sTGC strip-clusters}\label{fig:pitchfit}
	\end{subfigure}
	\hfill
	\begin{subfigure}[b]{0.45\textwidth}
		\centering
		\includegraphics[width=\textwidth]{strip-aligned.png}
		\caption{The differential after sinusoidal correction is applied}\label{fig:pitchaligned}
	\end{subfigure}
	\hfill
	\caption{Charge distribution over strip-pitch}\label{fig:strip-pitch}
\end{figure}


The charge measured on the strips of the sTGC detector results from a spatial sampling and discretization of the induced
charge. The process of reconstructing the sTGC strip-cluster position from this sampling introduces a differential
non-linearity effect on the reconstructed strip-cluster position. The deviation of the measured strip-cluster position
from the expected position (estimated by the pixel telescope track) depends on the strip-cluster position relative to
the strips.
This dependence is clearly seen in the two dimensional distributions in Fig.\ref{fig:pitchfit}. It shows the y-residual versus
strip-cluster position relative to the closest inter-strip gap center $y_{sTGC,0}^{rel}$. This effect is corrected using
a sinusoidal function.
\begin{equation}
 y_{sTGC} = y_{sTGC,0}-a_i\sin \left(2\pi y_{sTGC,0}^{rel}\right)
\end{equation}
where $y_{sTGC,0}$ is the strip-cluster mean resulting from the Gaussian fit and $y_{sTGC}$ is the corrected particle
position estimator. The amplitude parameters are denoted $a_i$ for the 3,4 and 5 strip-mutiplicity (cluster size). These
amplitude parameters are free parameters in the fit. The values of the amplitude parameters obtained from the fit to
data are compatible with being equal for the three strip-cluster multiplicity as shown in table \ref{table}.\par

\begin{table}
	\centering
	\caption{Fit parameters per cluster size}\label{table}
	\begin{tabular}{cc}
	\hline
	Strip-cluster multiplicity $i$ & Amplitude parameter $a_i$\\
	\hline
	3 & 205 $\pm$ 9\\
	4 & 206 $\pm$ 4\\
	5 & 211 $\pm$ 5\\
	\hline
	\end{tabular}
\end{table}

The correction function is therefore universal and is shown in Fig.\ref{fig:pitchfit}. The two-dimensional
distrubition after the correction is applied was to be reasonably flat as shown in Fig.\ref{fig:pitchaligned}.\par
The alignment of the coordinate system of the pixel telescope with respecto to the above-defined coordinate system of
the sTGC layer also affects the measured residual distribution. A simple two-paramter model is used to account for
translations and rotations of the two coordinate systems with respecto to each other. Both the alignment correction and
the differential non-linearity correction are included {\it in situ} in the analisys. The alignment correction is
introduced in the model by expressing the pixel track position in the sTGC-layer coordinate system $y_{pix}$, is a
function of the track position in the pixel telescope coordinate system $x_{pix}'$ and $y_{pix}'$, and two misalignment
parameters $\delta y$ and $\phi_{xy}$, as follows:
\begin{equation}
y_{pix}=-x_{pix}' \sin \phi_{xy} + y_{pix}'\cos \phi_{xy}+\delta y
\end{equation}
The variable $\delta y$ corresponds to a misalignment along the y-axis of the sTGC coordinate system, and $\phi_{xy}$
corresponds to a rotation of the telescope coordinate system in the x-y plane around the z-axis of the sTGC coordinate
system. Translation and rotation misalignment along and around the other axis are not taken into account in this model,
since they are expected to have a small impact on the determination of the intrinsic position resolution.\par


\begin{figure}[ht]
	\centering
	\hfill
	\begin{subfigure}[b]{0.42\textwidth}
		\centering
		\includegraphics[width=\textwidth]{y-xplane.png}
		\caption{Residual y against x position on pixel telescope.}\label{xyplanefit}
	\end{subfigure}
	\hfill
	\begin{subfigure}[b]{0.42\textwidth}
		\centering
		\includegraphics[width=\textwidth]{y-xplanerotate.png}
		\caption{After rotation and translation applied}\label{xrotate}
	\end{subfigure}
	\hfill
	\caption{Coordinate system correction}\label{rotation}
\end{figure}
On the figure\ref{xyplanefit} is possible to observe the y-residual mean increase linearly as a function of the x
position on the telescope called x$'_{pix}$, which is evidence for a small rotation between the two coordinate systems. 
The red line represents the correction applied to this dataset. Accounting for this correction results in a distribution
that is independent of x$'_{pix}$ on figure \ref{xrotate}.\par
After all the corrections applied, the calculations for the intrinsice resolutions are taking each layer from the sTGC
and comparing the residual distribution, and fitted by a double Gaussian function, where the first one represent the core
of the residual distrubution and the second is a wider Gaussian which represent some reconstructed strip-cluster from
background sources.
\begin{eqnarray*}
F_i &=& F_i(y_{sTGC,0}, y_{sTGC,0}^{rel}, x'_{pix},y'_{pix};\delta y,\phi_{xy},a_i,\sigma,f,\sigma_w)\\
		&=& f G(y_{sTGC}-y_{pix};0,\sigma) + (1-f)G(y_{sTGC}-y_{pix};0,\sigma_w)
\end{eqnarray*}
\begin{figure}[ht]
\centering
\includegraphics[width=0.4\textwidth]{resTelescope.png}
\caption{Intrinsic resolution of layer 4, respect to pixel telescope.}\label{restelescope}
\end{figure}

\begin{figure}[ht]
\centering
\hfill
\caption{Summary of pixel telescope analysis}\label{summary}
\begin{subfigure}[b]{0.3\textwidth}
\includegraphics[width=\textwidth]{beampos.png}
\caption{Beam position for different data sets on sTGC}\label{beampos}
\end{subfigure}
\hfill
\begin{subfigure}[b]{0.6\textwidth}
\centering
\includegraphics[width=\textwidth]{summaryTelescope.png}
\caption{Summary of intrisic resolution for each data set and layer of sTGC}\label{}
\end{subfigure}
\hfill
\end{figure}


On figure\ref{restelescope} a set of events shows the distribution presented before with a intrinsic resolution
parameter $\sigma$ of about
\unit{44}{\micro m} for a representative data taking run and sTGC strip-layer, where on red is the narrow Gaussian fit
and on braking red line the wider Gaussian fit. The fraction of the data parameterized by the narrow Gaussian and it is
typically around 95\% with a RMS of about 2\%. The rest of data taking runs and its beam position can be observe on
figure \ref{summary}, where the black circles represent the valid data and the open circles the runs with expected
degration due detector structure support or mis-calibrations.\par




%----------------- STANDALONE ANALYSIS 	---------------


\subsubsection{sTGC standalone analysis}
\begin{figure}[ht]
\centering
\begin{subfigure}[b]{0.45\textwidth}
\centering
\includegraphics[width=\textwidth]{resStandalone.png}
\caption{}\label{pairwise}
\end{subfigure}
\begin{subfigure}[b]{0.45\textwidth}
\centering
\includegraphics[width=\textwidth]{summaryStandalone.png}
\caption{}\label{}
\end{subfigure}
\end{figure}

In this analysis the correction for the differential non-linearity respect to strip-ptich obtained before is keeped,
however the residual distribution of the y-position is calculated from two pairwise layer of the sTGC, therefore half of
the variance of this distribution correspond to our parameter to estimate the intrinsic resolution for one layer, hence
$\sigma = \sigma_{residual}/\sqrt{2}$ and a strip-layer position residual distribution for a representative sTGC
standalone data taking run is shown in figure \ref{pairwise}. In this graph, a intrinsic resolution of $\sigma=40.8\pm0.8\mu m$ is obtained.\par 
In summary for fourteen data sets, the intrinsic resolution with this analysis is about \unit{45}{\micro m}. The white open
circles on the graph \ref{summary} correspond to non-validate data due to wire-support position or mis-calibrations. The
hash band represent the RMS spread and the blue line the average.\par 



\section{Pad efficiency}
One of the new features of the small-strip Thin Gap Chamber compare to its previous version is the posibility to provide
a fast trigger for the Region of Interest from the \unit[8x50]{cm$^2$} pad area, where 3 out of 4 pads from a sTGC
quadruaplet can confirm a particle candidate, therefore a track position can be obtained from the strips within this
area.\par
A test beam experiment was conducted at the CERN H6 facilities, using a \unit[130]{GeV} muon beam of about
\unit[4]{cm}
radius, a wider beam spot to test the characteristics of the pads. The setup is shown in figure \ref{padsetup} where the system was
triggered by a set of scintillators (in blue) with a \unit[12x12]{cm$^2$} coincidence area.\par
As explained before, for the beam tests a preliminary front-end electronics based on the VMM1 was used. This ASIC
provide a Time-over-threshold signal as digital output, however is also possible to get a analog pulses.
During the test beam, using the present configuration an inefficiency was observed related to small late charges from
the sTGC detector which may be not well adapted to the VMM1. An efficiency of 80-90\% was observe running at
\unit[100]{kHz/cm$^2$}.\par


\begin{figure}[ht]
\centering
\includegraphics[width=0.6 \textwidth]{padSetup.png}
\caption{Setup for pad measurments, coincidence block on ligth green.}\label{padsetup}
\end{figure}


To ensure that no inefficiency was due the the detector itself, the large cathode pads were used to estimate the detector
efficiency, which was measured by looking at the analog output of the fron-end amplifier. The efficiency of the pad $n$
in the first layer was defined with respect to the coincidence of the trigger with a signal in the fully overlapping pad
of the second layer. \par



\begin{figure}[ht]
\centering
\begin{subfigure}[b]{0.45\textwidth}
\centering
\includegraphics[width=\textwidth]{padnormal.png}
\caption{ToT signal from pad L1, pad L2 as a trigger}\label{scope1}
\end{subfigure}
\begin{subfigure}[b]{0.45\textwidth}
\centering
\includegraphics[width=\textwidth]{paddeadtime.png}
\caption{Large dead time on ToT signal from pad L1}\label{scope2}
\end{subfigure}
\caption{Digital and analog signal from VMM1}\label{scope}
\end{figure}

Two examples from this configurations is shown on figure \ref{scope}, on the left a two analog signals from pads are
present with a ToT signal from layer 1 with about 2 \micro s length, meanwhile on the right picture a long ToT pulse with more
than 40 \micro s length when the two analog signal are present. By recording hundreds of triggered events using an
oscilloscope, the presence of a detector signal whithin the live-part of the front-end electronics (independent of the
signal threshold) was checked. This test confirmed that the detector was 100\% efficient.\par


\subsubsection{charge sharing between pads}
\begin{figure}[ht]
\centering
\includegraphics[width=.5\textwidth]{chargepos.png}
\caption{Pad region transition scheme}\label{chargepos}
\end{figure}

In manner to study the transition region between pads, the scintillator coincidence triggering area and the particle
beam were centred between pad $n$ and pad $n+1$ of the first layer, as illustraded in Figure \ref{chargepos}.\par
After applying timming quality requirements on the strip and pad hits, the channel baseline values are subtracted from
the analog peak values. Strip-clusters with induced charge in either 3,4 or 5 adjacent strips are selected and
calibrated in the same way as for the Fermilab beam test. Events with a single strip-cluster in the first layer and the
second layer are selected. The strip-cluster position (mean of the fitted Gaussian) in the first layer is used to define
the position of the particle going through the detector. The events are further required to contain a hit above
threshold on either pad $n$ or pad $n+1$. The charge fraction ($F$) is defined using the analog peak values ($P$) of the two
adjacent pads:
\begin{equation}
F = \frac{P_n - P_{n+1}}{P_n + P_{n+1}}
\end{equation}

\begin{figure}[ht]
\centering
\includegraphics[width=0.5\textwidth]{chargeSharing.png}
\caption{Charge Sharing distritbution}\label{chargeSharing}
\end{figure}

Figure \ref{chargeSharing} shows the charge fraction as a function of the position with respect to the centre of the
transition region, where the two pads share more than 70\% of the induced charge, spans about \unit{4}{mm}.








\section{Summary}
Simulations, construction , calibrations, characterization of sTGC has been done. 
What's next? Test with cosmic rays for efficiency measurements, and test under hadron and muon beams with NSW electronic
readout. 
