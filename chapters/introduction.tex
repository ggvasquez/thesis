
In this chapter we presente a review of the physcis for radiation detectors.
The basic features of radiation detectors can be understood once fundamental processes of radiation interaction in
matter are considered.
\section{Interaccion of Radiation with Matter}
5 pages
\section{Gas fillled detectors}
5 pages
\subsection{Production of Electron-Ion pairs}
\subsection{Diffusion and Drift of Charges in Gases}
\subsection{Regions of Operation of Gas Filled Detectors}
\section{Scintillation detectors}
3 pages
\section{Calorimeters}
\subsection{Radiation Length}
The radiation length is the distance over which the radiative emission is the dominant energyloss process and the
screening parameter $\eta$ approaches to 0, the total radiation cross section is that for complete screening except in
the case of high frequency emitted photons. This cross section does not depend on the incoming electron energy $E_0$.
For a complete screening in the Born approximation, the quantity $X_0$ is introduced as:
\begin{equation}\label{x0}
X_0 = \frac{1}{\left[4n_A\bar{\Phi}_c\ln\left(\frac{183}{Z^{1/3}}\right)\right]}\mathrm{[cm]}
\end{equation}

4 pages




